\documentclass[11pt,a4paper]{article}
\usepackage[utf8]{inputenc}
\usepackage[T1]{fontenc}
\usepackage{geometry}
\usepackage{hyperref}
\usepackage{booktabs}
\usepackage{xcolor}
\usepackage{fancyhdr}
\usepackage{listings}

\geometry{margin=2.5cm}

\definecolor{refblue}{RGB}{0,82,147}
\definecolor{codebg}{RGB}{245,245,245}

\lstset{
    backgroundcolor=\color{codebg},
    basicstyle=\ttfamily\small,
    breaklines=true,
    frame=single,
    rulecolor=\color{gray},
}

\hypersetup{
    colorlinks=true,
    linkcolor=refblue,
    urlcolor=refblue,
    pdftitle={REF-Manager Quick Start Guide},
    pdfauthor={George Tsoulas},
}

\pagestyle{fancy}
\fancyhf{}
\fancyhead[L]{\textit{REF-Manager Quick Start}}
\fancyhead[R]{\textit{Version 4.0}}
\fancyfoot[C]{\thepage}

\title{
    {\Huge\bfseries\color{refblue} REF-Manager}\\[0.3cm]
    {\LARGE Quick Start Guide}\\[0.3cm]
    {\large Get Running in 15 Minutes}\\[0.5cm]
    {\large Version 4.0.0}
}
\author{George Tsoulas\\University of York}
\date{December 2025}

\begin{document}

\maketitle

\section{Prerequisites}

Before starting, ensure you have:
\begin{itemize}
    \item Python 3.10 or higher installed
    \item Git installed
    \item Terminal/command line access
\end{itemize}

\section{Step 1: Download and Setup (5 minutes)}

\begin{lstlisting}
# Clone the repository
git clone https://github.com/gtsoulas/ref-manager.git
cd ref-manager

# Create virtual environment
python -m venv venv

# Activate it
source venv/bin/activate   # Linux/Mac

# Install dependencies
pip install -r requirements.txt
pip install requests
\end{lstlisting}

\section{Step 2: Configure (2 minutes)}

\begin{lstlisting}
# Copy example configuration
cp env.example .env
\end{lstlisting}

For development, the defaults work fine.

\section{Step 3: Initialise Database (3 minutes)}

\begin{lstlisting}
# Create database tables
python manage.py migrate

# Create admin account
python manage.py createsuperuser

# Setup user roles
python manage.py setup_roles --create-profiles --superusers-admin
\end{lstlisting}

\section{Step 4: Start the Server (1 minute)}

\begin{lstlisting}
python manage.py runserver
\end{lstlisting}

Visit: \textbf{http://localhost:8000}

\section{Step 5: First Steps (4 minutes)}

\subsection{Add an Output with DOI Auto-Fetch}

\begin{enumerate}
    \item Navigate to \textbf{Outputs $\rightarrow$ Add Output}
    \item Paste a DOI (e.g., \texttt{10.1038/nature12373})
    \item Click \textbf{Auto-Fill}
    \item Review populated fields
    \item Select the Colleague
    \item Add O/S/R ratings (0.00-4.00)
    \item Click \textbf{Save}
\end{enumerate}

\section{O/S/R Rating Scale}

\begin{table}[h]
\centering
\begin{tabular}{ccl}
\toprule
\textbf{Score} & \textbf{Star} & \textbf{Description} \\
\midrule
3.50 - 4.00 & 4$\star$ & World-leading \\
2.50 - 3.49 & 3$\star$ & Internationally excellent \\
1.50 - 2.49 & 2$\star$ & Internationally recognised \\
0.50 - 1.49 & 1$\star$ & Nationally recognised \\
0.00 - 0.49 & U & Unclassified \\
\bottomrule
\end{tabular}
\end{table}

\section{What's Next?}

\begin{itemize}
    \item \textbf{User Guide}: Complete feature documentation
    \item \textbf{Technical Documentation}: Server deployment
    \item \textbf{Troubleshooting}: Common issues
\end{itemize}

\section{Quick Commands}

\begin{table}[h]
\centering
\begin{tabular}{ll}
\toprule
\textbf{Task} & \textbf{Command} \\
\midrule
Start server & \texttt{python manage.py runserver} \\
Create migration & \texttt{python manage.py makemigrations} \\
Apply migrations & \texttt{python manage.py migrate} \\
Collect static & \texttt{python manage.py collectstatic} \\
\bottomrule
\end{tabular}
\end{table}

\section{Need Help?}

\begin{itemize}
    \item Email: george.tsoulas@york.ac.uk
    \item GitHub: https://github.com/gtsoulas/ref-manager
\end{itemize}

\vfill
\begin{center}
{\small REF-Manager v4.0.0 -- George Tsoulas, University of York}
\end{center}

\end{document}
