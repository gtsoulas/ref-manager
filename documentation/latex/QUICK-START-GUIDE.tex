\documentclass[11pt,a4paper]{article}
\usepackage[utf8]{inputenc}
\usepackage[margin=2.5cm]{geometry}
\usepackage{hyperref}
\usepackage{xcolor}
\usepackage{listings}
\usepackage{tcolorbox}
\usepackage{fontawesome5}
\usepackage{tikz}

% Colors
\definecolor{codegreen}{rgb}{0,0.6,0}
\definecolor{codegray}{rgb}{0.5,0.5,0.5}
\definecolor{codepurple}{rgb}{0.58,0,0.82}
\definecolor{backcolour}{rgb}{0.95,0.95,0.92}
\definecolor{primaryblue}{RGB}{0,102,204}

% Code listing style
\lstdefinestyle{mystyle}{
    backgroundcolor=\color{backcolour},
    commentstyle=\color{codegreen},
    keywordstyle=\color{magenta},
    numberstyle=\tiny\color{codegray},
    stringstyle=\color{codepurple},
    basicstyle=\ttfamily\small,
    breakatwhitespace=false,
    breaklines=true,
    captionpos=b,
    keepspaces=true,
    numbers=left,
    numbersep=5pt,
    showspaces=false,
    showstringspaces=false,
    showtabs=false,
    tabsize=2
}
\lstset{style=mystyle}

% Hyperref setup
\hypersetup{
    colorlinks=true,
    linkcolor=blue,
    filecolor=magenta,
    urlcolor=cyan,
    pdftitle={REF Manager v2.0 - Quick Start Guide},
    pdfauthor={George Tsoulos},
}

% Custom commands
\newcommand{\cmd}[1]{\texttt{#1}}
\newcommand{\file}[1]{\texttt{#1}}

\title{\textbf{REF Manager v2.0}\\
\Large Quick Start Guide\\
\large Get up and running in 10 minutes!}
\author{George Tsoulas\\
Department of Language and Linguistic Science\\
University of York}
\date{November 3, 2025\\
Version 2.0.0}

\begin{document}

\maketitle
\thispagestyle{empty}

\begin{abstract}
This Quick Start Guide will help you install and configure REF Manager v2.0 in approximately 10 minutes. Whether you're setting up for the first time or upgrading from version 1.0, this guide provides step-by-step instructions to get your system running quickly.
\end{abstract}

\newpage
\tableofcontents
\newpage

\section{Prerequisites}

Before you start, ensure you have:

\begin{itemize}
    \item Python 3.10 or higher installed
    \item Terminal/Command Prompt access
    \item Internet connection
    \item 500MB free disk space
\end{itemize}

\subsection{Quick Check}

Verify your Python version:

\begin{lstlisting}[language=bash]
python3 --version
\end{lstlisting}

You should see version 3.10 or higher.

\section{Installation (10 Minutes)}

\subsection{Step 1: Download and Setup (2 minutes)}

\begin{lstlisting}[language=bash]
# Create project directory
mkdir -p ~/ref-project-app
cd ~/ref-project-app

# If using Git
git clone https://github.com/yourusername/ref-manager.git
cd ref-manager
\end{lstlisting}

Or extract ZIP file if downloaded manually.

\subsection{Step 2: Create Virtual Environment (1 minute)}

\begin{lstlisting}[language=bash]
# Create virtual environment
python3 -m venv venv

# Activate it
# Linux/macOS:
source venv/bin/activate

# Windows:
venv\Scripts\activate

# You should see (venv) in your prompt
\end{lstlisting}

\subsection{Step 3: Install Dependencies (3 minutes)}

\begin{lstlisting}[language=bash]
# Upgrade pip
pip install --upgrade pip

# Install all required packages
pip install -r requirements.txt

# If on Python 3.13 and getting errors:
pip install --break-system-packages -r requirements.txt
\end{lstlisting}

\textbf{What gets installed:}
\begin{itemize}
    \item Django 4.2+ (web framework)
    \item django-crispy-forms (beautiful forms)
    \item openpyxl (Excel export) \textbf{NEW in v2.0}
    \item gunicorn (production server)
    \item psycopg2-binary (PostgreSQL support)
\end{itemize}

\subsection{Step 4: Configure Environment (1 minute)}

\begin{lstlisting}[language=bash]
# Copy environment template
cp .env.example .env

# Generate a secret key
python -c "from django.core.management.utils import get_random_secret_key; print(get_random_secret_key())"

# Edit .env file
nano .env  # or use any text editor
\end{lstlisting}

\textbf{Minimum .env configuration:}

\begin{lstlisting}
SECRET_KEY=paste-the-secret-key-here
DEBUG=True
ALLOWED_HOSTS=localhost,127.0.0.1
\end{lstlisting}

\subsection{Step 5: Initialize Database (2 minutes)}

\begin{lstlisting}[language=bash]
# Create database tables
python manage.py migrate

# Create admin user
python manage.py createsuperuser
# Enter username, email, and password when prompted
\end{lstlisting}

\subsection{Step 6: Start the Server (1 minute)}

\begin{lstlisting}[language=bash]
# Run development server
python manage.py runserver

# Server should start at http://127.0.0.1:8000
\end{lstlisting}

\begin{tcolorbox}[colback=green!5!white,colframe=green!75!black,title=\textbf{Congratulations!}]
Your REF Manager is now running! Open your browser and go to:\\
\url{http://localhost:8000}
\end{tcolorbox}

\newpage

\section{First Steps After Installation}

\subsection{Log In (1 minute)}

\begin{enumerate}
    \item Open \url{http://localhost:8000}
    \item Click \textbf{``Login''} in the top right
    \item Enter your superuser credentials
    \item You'll see the dashboard
\end{enumerate}

\subsection{Explore the Dashboard (2 minutes)}

The dashboard shows:
\begin{itemize}
    \item \textbf{Submission Statistics}: Output counts and quality profile
    \item \textbf{Staff Summary}: Current and former colleagues
    \item \textbf{Review Progress}: Critical Friends and Internal Panel
    \item \textbf{Task Overview}: \textbf{NEW} - Urgent and overdue tasks
    \item \textbf{Recent Activity}: Latest changes
\end{itemize}

\subsection{Add Your First Colleague (3 minutes)}

\begin{enumerate}
    \item Click \textbf{``Colleagues''} in navigation
    \item Click \textbf{``Add Colleague''} button
    \item Fill in:
    \begin{itemize}
        \item First name, Last name
        \item Email
        \item Unit of Assessment
        \item FTE (Full-Time Equivalent, e.g., 1.0)
        \item \textbf{Employment Status}: Current \textbf{(NEW in v2.0)}
        \item \textbf{Category}: Choose appropriate type \textbf{(NEW in v2.0)}
    \end{itemize}
    \item Click \textbf{``Save''}
\end{enumerate}

\begin{tcolorbox}[colback=blue!5!white,colframe=blue!75!black,title=\textbf{NEW Categories in v2.0}]
\begin{itemize}
    \item Independent Researcher
    \item Non-Independent Researcher (for post-docs!)
    \item Post-Doctoral Researcher
    \item Research Assistant
    \item Academic Staff
    \item Support Staff
    \item Co-author (External)
\end{itemize}
\end{tcolorbox}

\newpage

\section{Try v2.0 Features (5 minutes)}

\subsection{Create a Task}

\begin{enumerate}
    \item Click \textbf{``Tasks''} in navigation
    \item Click \textbf{``Create Task''}
    \item Fill in:
    \begin{itemize}
        \item Title
        \item Category (e.g., ``Submission'', ``Administrative'')
        \item Priority (Low, Medium, High, Urgent)
        \item Due Date
        \item Description
    \end{itemize}
    \item Click \textbf{``Save''}
\end{enumerate}

\subsection{Set Up Internal Panel}

\begin{enumerate}
    \item Click \textbf{``Internal Panel''} in navigation
    \item Click \textbf{``Add Panel Member''}
    \item Select a colleague
    \item Choose role (Chair, Member, Specialist)
    \item Click \textbf{``Save''}
\end{enumerate}

\subsection{Try CSV Import}

\begin{enumerate}
    \item Click \textbf{``Outputs''} $\rightarrow$ \textbf{``Import''}
    \item Download the CSV template
    \item Fill in output data
    \item Upload CSV file
    \item Review and confirm import
\end{enumerate}

\newpage

\section{Daily Workflow}

\subsection{Typical REF Manager Day}

\subsubsection{Morning (5 minutes)}
\begin{itemize}
    \item Check dashboard for urgent tasks
    \item Review overdue reviews
    \item Check recent activity
\end{itemize}

\subsubsection{During the Day}
\begin{itemize}
    \item Add new outputs as they're published
    \item Update colleague information
    \item Assign papers for review
    \item Track review progress
    \item Complete tasks
\end{itemize}

\subsubsection{Weekly (15 minutes)}
\begin{itemize}
    \item Review quality profile
    \item Check submission progress
    \item Update request status
    \item Generate reports
\end{itemize}

\subsubsection{Monthly (30 minutes)}
\begin{itemize}
    \item Full review progress audit
    \item Update staff status (current/former)
    \item Export data for meetings
    \item Generate comprehensive reports
\end{itemize}

\newpage

\section{Running as Background Service}

\subsection{Option 1: Using Screen (Simplest)}

\begin{lstlisting}[language=bash]
# Install screen
sudo apt-get install screen

# Start screen session
screen -S ref-manager

# Run server
cd ~/ref-project-app/ref-manager
source venv/bin/activate
python manage.py runserver 0.0.0.0:8000

# Detach: Press Ctrl+A, then D

# Reattach later
screen -r ref-manager
\end{lstlisting}

\subsection{Option 2: Using systemd (Production)}

Create service file:

\begin{lstlisting}[language=bash]
sudo nano /etc/systemd/system/ref-manager.service
\end{lstlisting}

Add this content (replace \cmd{YOUR\_USERNAME}):

\begin{lstlisting}
[Unit]
Description=REF Manager Django Application
After=network.target

[Service]
Type=simple
User=YOUR_USERNAME
WorkingDirectory=/home/YOUR_USERNAME/ref-project-app/ref-manager
Environment="PATH=/home/YOUR_USERNAME/ref-project-app/ref-manager/venv/bin"
ExecStart=/home/YOUR_USERNAME/ref-project-app/ref-manager/venv/bin/gunicorn \
    --workers 3 \
    --bind 0.0.0.0:8000 \
    ref_manager.wsgi:application
Restart=always

[Install]
WantedBy=multi-user.target
\end{lstlisting}

Enable and start:

\begin{lstlisting}[language=bash]
sudo systemctl enable ref-manager
sudo systemctl start ref-manager
sudo systemctl status ref-manager
\end{lstlisting}

\subsection{Manage the Service}

\begin{lstlisting}[language=bash]
# Start
sudo systemctl start ref-manager

# Stop
sudo systemctl stop ref-manager

# Restart (after code changes)
sudo systemctl restart ref-manager

# Check status
sudo systemctl status ref-manager

# View logs
sudo journalctl -u ref-manager -f
\end{lstlisting}

\newpage

\section{Quick Commands Reference}

\subsection{Daily Commands}

\begin{lstlisting}[language=bash]
# Start server (development)
python manage.py runserver

# Start server (accessible from other computers)
python manage.py runserver 0.0.0.0:8000

# Access Django shell
python manage.py shell

# Create backup
cp db.sqlite3 backups/db.sqlite3.$(date +%Y%m%d)
\end{lstlisting}

\subsection{Admin Tasks}

\begin{lstlisting}[language=bash]
# Create new user
python manage.py createsuperuser

# Change password
python manage.py changepassword username

# Collect static files (production)
python manage.py collectstatic --noinput
\end{lstlisting}

\subsection{Database Tasks}

\begin{lstlisting}[language=bash]
# Create migrations (after model changes)
python manage.py makemigrations

# Apply migrations
python manage.py migrate

# Show migration status
python manage.py showmigrations

# Access database directly
python manage.py dbshell
\end{lstlisting}

\newpage

\section{Quick Fixes for Common Issues}

\subsection{Can't Log In}

\begin{lstlisting}[language=bash]
# Reset password
python manage.py changepassword yourusername
\end{lstlisting}

\subsection{Page Not Loading / CSS Not Working}

\begin{lstlisting}[language=bash]
# Collect static files
python manage.py collectstatic --noinput

# Restart server
# Press Ctrl+C, then run again
python manage.py runserver
\end{lstlisting}

\subsection{``No Such Table'' Errors}

\begin{lstlisting}[language=bash]
# Run migrations
python manage.py migrate
\end{lstlisting}

\subsection{Python 3.13 Compatibility}

If you see: \cmd{TypeError: unsupported operand type(s) for *: 'decimal.Decimal' and 'float'}

This is already fixed in v2.0 code. Update to latest version.

\newpage

\section{What's New in v2.0}

\subsection{Major Additions}

\begin{itemize}
    \item[\faCheck] \textbf{Employment Status Tracking} - Current vs Former staff
    \item[\faCheck] \textbf{Enhanced Categories} - 9 colleague types
    \item[\faCheck] \textbf{Internal Panel System} - Internal reviewer management
    \item[\faCheck] \textbf{Task Management} - Track all activities
    \item[\faCheck] \textbf{CSV Import} - Bulk operations
    \item[\faCheck] \textbf{Excel Export} - Export with clickable links
    \item[\faCheck] \textbf{Enhanced Requests} - Mark complete, delete
    \item[\faCheck] \textbf{Dashboard Updates} - New widgets
\end{itemize}

\section{Next Steps}

After Quick Start:

\begin{enumerate}
    \item \textbf{Read the User Guide} (\file{USER-GUIDE.md})
    \begin{itemize}
        \item Detailed feature documentation
        \item Best practices
        \item Advanced workflows
    \end{itemize}
    
    \item \textbf{Check Technical Documentation} (\file{TECHNICAL-DOCUMENTATION.md})
    \begin{itemize}
        \item System architecture
        \item Development guide
        \item API reference
    \end{itemize}
    
    \item \textbf{Review Troubleshooting Guide} (\file{TROUBLESHOOTING.md})
    \begin{itemize}
        \item Detailed problem solving
        \item Performance optimization
        \item Security considerations
    \end{itemize}
    
    \item \textbf{Read the Changelog} (\file{CHANGELOG.md})
    \begin{itemize}
        \item Version history
        \item All changes and improvements
        \item Migration notes
    \end{itemize}
\end{enumerate}

\section{Pro Tips}

\begin{enumerate}
    \item \textbf{Regular Backups}: Back up database daily
    \begin{lstlisting}[language=bash]
cp db.sqlite3 backups/db.sqlite3.$(date +%Y%m%d)
    \end{lstlisting}
    
    \item \textbf{Use Categories}: Properly categorize colleagues for better reporting
    
    \item \textbf{Track Everything}: Use tasks to track all REF activities
    
    \item \textbf{Export Often}: Export data regularly for meetings and reports
    
    \item \textbf{Former Staff}: Don't delete former staff - mark as ``Former'' instead
    
    \item \textbf{CSV Import}: Use CSV import for bulk data entry - much faster!
    
    \item \textbf{Excel Export}: Export assignments with links to send to reviewers
    
    \item \textbf{Dashboard}: Check dashboard daily for urgent items
    
    \item \textbf{Internal vs External}: Use Internal Panel for university reviewers, Critical Friends for external
    
    \item \textbf{Quality Ratings}: Update ratings as papers are evaluated
\end{enumerate}

\section{Checklist}

Use this checklist to ensure proper setup:

\subsection{Installation}
\begin{itemize}
    \item[\faSquare] Python 3.10+ installed
    \item[\faSquare] Virtual environment created
    \item[\faSquare] Dependencies installed
    \item[\faSquare] Environment variables configured
    \item[\faSquare] Database initialized
    \item[\faSquare] Superuser created
    \item[\faSquare] Server runs without errors
\end{itemize}

\subsection{Initial Setup}
\begin{itemize}
    \item[\faSquare] First login successful
    \item[\faSquare] Dashboard displays correctly
    \item[\faSquare] Added at least one colleague
    \item[\faSquare] Added at least one output
    \item[\faSquare] Created a task (v2.0)
    \item[\faSquare] Set up Internal Panel member (v2.0)
\end{itemize}

\subsection{Testing v2.0 Features}
\begin{itemize}
    \item[\faSquare] Tried colleague categories
    \item[\faSquare] Marked someone as former staff
    \item[\faSquare] Created an internal panel member
    \item[\faSquare] Created and completed a task
    \item[\faSquare] Tested CSV import
    \item[\faSquare] Tested Excel export with links
\end{itemize}

\subsection{Production (Optional)}
\begin{itemize}
    \item[\faSquare] Background service configured
    \item[\faSquare] Automatic startup enabled
    \item[\faSquare] Backup script created
    \item[\faSquare] Staff trained
\end{itemize}

\vspace{1cm}

\begin{tcolorbox}[colback=blue!5!white,colframe=blue!75!black,title=\textbf{Need More Help?}]
See the complete documentation suite or contact your system administrator.\\[0.5cm]
\textbf{Ready to become a REF Manager pro?} Read the User Guide next!
\end{tcolorbox}

\vfill

\begin{center}
\rule{0.8\textwidth}{0.4pt}\\
\textbf{Version:} 2.0.0 \hspace{1cm} \textbf{Last Updated:} November 3, 2025\\
\textbf{Prepared by:} George Tsoulas\\
\textbf{Institution:} Department of Language and Linguistic Science, University of York
\end{center}

\end{document}
