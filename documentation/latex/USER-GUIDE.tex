\documentclass[11pt,a4paper]{article}
\usepackage[utf8]{inputenc}
\usepackage[T1]{fontenc}
\usepackage{geometry}
\usepackage{graphicx}
\usepackage{hyperref}
\usepackage{booktabs}
\usepackage{longtable}
\usepackage{xcolor}
\usepackage{fancyhdr}
\usepackage{titlesec}
\usepackage{enumitem}
\usepackage{tcolorbox}

\geometry{margin=2.5cm}

\definecolor{refblue}{RGB}{0,82,147}
\definecolor{refgreen}{RGB}{40,167,69}
\definecolor{refred}{RGB}{220,53,69}
\definecolor{refyellow}{RGB}{255,193,7}

\hypersetup{
    colorlinks=true,
    linkcolor=refblue,
    urlcolor=refblue,
    pdftitle={REF-Manager User Guide},
    pdfauthor={University of York}
}

\pagestyle{fancy}
\fancyhf{}
\fancyhead[L]{\textit{REF-Manager User Guide}}
\fancyhead[R]{\textit{Version 3.0}}
\fancyfoot[C]{\thepage}

\titleformat{\section}{\Large\bfseries\color{refblue}}{\thesection}{1em}{}
\titleformat{\subsection}{\large\bfseries\color{refblue}}{\thesubsection}{1em}{}

\title{
    \vspace{-2cm}
    {\Huge\bfseries\color{refblue} REF-Manager}\\[0.5cm]
    {\LARGE User Guide}\\[0.3cm]
    {\large Version 3.0.0}\\[0.3cm]
    {\normalsize Research Excellence Framework Submission Management}
}
\author{Department of Language and Linguistic Science\\University of York}
\date{November 2024}

\begin{document}

\maketitle
\thispagestyle{empty}

\begin{abstract}
This User Guide provides comprehensive documentation for REF-Manager, a web-based application designed to help UK university departments manage their Research Excellence Framework (REF) submissions. The guide covers all features including colleague and output management, quality assessments, risk assessment, portfolio optimisation, and reporting.
\end{abstract}

\tableofcontents
\newpage

% ==========================================
\section{Getting Started}
% ==========================================

\subsection{Logging In}

\begin{enumerate}
    \item Navigate to your REF-Manager URL
    \item Enter your username and password
    \item Click \textbf{Log In}
\end{enumerate}

\subsection{Understanding Your Role}

REF-Manager uses role-based access control with four distinct roles:

\begin{table}[h]
\centering
\begin{tabular}{lll}
\toprule
\textbf{Role} & \textbf{Description} & \textbf{Capabilities} \\
\midrule
Administrator & Full system access & All features \\
Observer & Read-only access & View all, export \\
Internal Panel & Departmental reviewer & Rate assigned outputs \\
Colleague & Staff member & Manage own outputs \\
\bottomrule
\end{tabular}
\caption{User roles and their capabilities}
\end{table}

Users may have multiple roles with combined permissions.

\subsection{Navigation}

The main menu provides access to:
\begin{itemize}
    \item \textbf{Dashboard}: Overview and statistics
    \item \textbf{Colleagues}: Staff management
    \item \textbf{Outputs}: Research output management
    \item \textbf{Critical Friends}: External reviewer management
    \item \textbf{Internal Panel}: Internal reviewer management
    \item \textbf{Reports}: Analytics and export
    \item \textbf{Tasks}: Task tracking
    \item \textbf{Admin}: System administration (Administrators only)
\end{itemize}

% ==========================================
\section{Dashboard}
% ==========================================

The dashboard provides an at-a-glance overview of your REF preparation status.

\subsection{Statistics Cards}

The dashboard displays key metrics:
\begin{itemize}
    \item \textbf{Total Colleagues}: Returnable staff count
    \item \textbf{Total Outputs}: All tracked outputs
    \item \textbf{Approved Outputs}: Ready for submission
    \item \textbf{In Review}: Currently being assessed
    \item \textbf{Pending Requests}: Outstanding items
\end{itemize}

\subsection{Quality Distribution}

A visual chart shows the distribution of outputs by quality rating:
\begin{itemize}
    \item \textcolor{refgreen}{\textbf{4★}}: World-leading
    \item \textcolor{refblue}{\textbf{3★}}: Internationally excellent
    \item \textcolor{refyellow}{\textbf{2★}}: Recognised internationally
    \item \textcolor{orange}{\textbf{1★}}: Recognised nationally
    \item \textbf{U}: Unclassified
\end{itemize}

% ==========================================
\section{Managing Colleagues}
% ==========================================

\subsection{Adding a Colleague}

\begin{enumerate}
    \item Navigate to \textbf{Colleagues}
    \item Click \textbf{Add Colleague}
    \item Complete the required fields:
    \begin{itemize}
        \item Staff ID (unique identifier)
        \item First and Last name
        \item FTE (0.1 to 1.0)
        \item Contract type
        \item Category
        \item Unit of Assessment
    \end{itemize}
    \item Click \textbf{Save}
\end{enumerate}

\subsection{Colleague Categories}

\begin{table}[h]
\centering
\begin{tabular}{ll}
\toprule
\textbf{Category} & \textbf{Description} \\
\midrule
Independent & Leads own research programme \\
Non-Independent & Contributes to others' research \\
Post-Doctoral & Post-doctoral researchers \\
Academic & Teaching-focused academic staff \\
Research Assistant & Research support roles \\
Support & Administrative/technical support \\
\bottomrule
\end{tabular}
\caption{Colleague categories}
\end{table}

\subsection{Required Outputs}

REF-Manager calculates required outputs based on FTE:
\begin{itemize}
    \item Formula: FTE $\times$ 2.5 (rounded down, maximum 5)
    \item Example: 0.8 FTE = 2 outputs required
\end{itemize}

% ==========================================
\section{Managing Outputs}
% ==========================================

\subsection{Adding an Output}

\begin{enumerate}
    \item Navigate to \textbf{Outputs}
    \item Click \textbf{Add Output}
    \item Complete the form:
    \begin{itemize}
        \item Colleague (select author)
        \item Title
        \item Publication Type
        \item Publication Year (2021-2028)
        \item Publication Venue
        \item Authors (all authors in citation format)
        \item Author Position
        \item Unit of Assessment
    \end{itemize}
    \item Click \textbf{Save}
\end{enumerate}

\subsection{Publication Types}

\begin{table}[h]
\centering
\begin{tabular}{cl}
\toprule
\textbf{Code} & \textbf{Type} \\
\midrule
A & Journal Article \\
B & Book \\
C & Book Chapter \\
D & Conference Paper \\
E & Patent \\
F & Software \\
G & Performance/Exhibition \\
H & Other \\
\bottomrule
\end{tabular}
\caption{Publication types}
\end{table}

\subsection{Output Status Workflow}

Outputs progress through the following statuses:

\begin{center}
Draft $\rightarrow$ Submitted $\rightarrow$ Internal Review $\rightarrow$ External Review $\rightarrow$ Approved
\end{center}

With alternative paths to Revision or Rejected as needed.

% ==========================================
\section{Quality Assessments}
% ==========================================

\subsection{Rating System}

REF-Manager uses the standard REF quality ratings:

\begin{table}[h]
\centering
\begin{tabular}{lll}
\toprule
\textbf{Rating} & \textbf{Description} & \textbf{GPA Value} \\
\midrule
4★ & World-leading quality & 4 \\
3★ & Internationally excellent & 3 \\
2★ & Recognised internationally & 2 \\
1★ & Recognised nationally & 1 \\
U & Unclassified & 0 \\
\bottomrule
\end{tabular}
\caption{Quality ratings}
\end{table}

\subsection{Multi-Dimensional Ratings}

Each output can receive three independent ratings:
\begin{enumerate}
    \item \textbf{Internal Panel Rating}: From departmental reviewers
    \item \textbf{Critical Friend Rating}: From external reviewers
    \item \textbf{Self-Assessment}: From the output author
\end{enumerate}

The average rating is automatically calculated from available assessments.

% ==========================================
\section{Risk Assessment}
% ==========================================

\subsection{Risk Score Components}

Each output has a composite risk score (0-1 scale):

\begin{table}[h]
\centering
\begin{tabular}{llll}
\toprule
\textbf{Score} & \textbf{Level} & \textbf{Colour} & \textbf{Action} \\
\midrule
0.00--0.24 & Low & Green & Proceed \\
0.25--0.49 & Medium-Low & Yellow & Monitor \\
0.50--0.74 & Medium-High & Orange & Mitigate \\
0.75--1.00 & High & Red & Urgent \\
\bottomrule
\end{tabular}
\caption{Risk levels}
\end{table}

\subsection{Risk Components}

\textbf{Content Risk (60\% weight)}
\begin{itemize}
    \item Panel disagreement likelihood
    \item Methodology concerns
    \item Approach controversies
\end{itemize}

\textbf{Timeline Risk (40\% weight)}
\begin{itemize}
    \item Based on publication status
    \item Published = 0.00, Planned = 1.00
\end{itemize}

\subsection{Open Access Compliance}

Open Access compliance is critical for REF eligibility:
\begin{itemize}
    \item Non-compliant outputs flagged with minimum 0.85 risk
    \item Check deposit dates and licences
    \item Verify compliance before submission
\end{itemize}

% ==========================================
\section{REF Submissions}
% ==========================================

\subsection{Creating a Submission Scenario}

\begin{enumerate}
    \item Navigate to \textbf{Reports $\rightarrow$ Submissions}
    \item Click \textbf{Create Submission}
    \item Enter name, UOA, and year
    \item Save
\end{enumerate}

\subsection{Submission Metrics}

Each submission calculates:
\begin{itemize}
    \item \textbf{Quality Score}: GPA-style average
    \item \textbf{Risk Score}: Portfolio risk level
    \item \textbf{Representativeness}: Research area coverage
    \item \textbf{Equality Score}: Staff inclusion percentage
    \item \textbf{Gender Balance}: Representation metric
\end{itemize}

% ==========================================
\section{Reports and Export}
% ==========================================

\subsection{Available Reports}

\begin{itemize}
    \item Submission Overview
    \item Quality Profile
    \item Staff Progress
    \item Review Status
    \item Risk Dashboard
    \item Comprehensive Report
\end{itemize}

\subsection{Export Formats}

\begin{itemize}
    \item \textbf{Excel}: Full data with formatting
    \item \textbf{CSV}: Simple tabular data
    \item \textbf{LaTeX}: Professional reports
    \item \textbf{JSON}: Risk analysis data
\end{itemize}

% ==========================================
\section{Task Management}
% ==========================================

\subsection{Creating Tasks}

\begin{enumerate}
    \item Navigate to \textbf{Tasks}
    \item Click \textbf{Create Task}
    \item Enter title, description, category, priority
    \item Assign to user and set due date
    \item Save
\end{enumerate}

\subsection{Task Priorities}

\begin{table}[h]
\centering
\begin{tabular}{ll}
\toprule
\textbf{Priority} & \textbf{Urgency} \\
\midrule
Low & When possible \\
Medium & This week \\
High & This week, prioritise \\
Urgent & Immediate \\
\bottomrule
\end{tabular}
\caption{Task priorities}
\end{table}

% ==========================================
\section{Getting Help}
% ==========================================

\subsection{Documentation}

Additional documentation available:
\begin{itemize}
    \item Quick Start Guide
    \item Technical Documentation
    \item Troubleshooting Guide
\end{itemize}

\subsection{Contact}

\begin{itemize}
    \item \textbf{Email}: george.tsoulas@york.ac.uk
    \item \textbf{GitHub}: https://github.com/gtsoulas/ref-manager
\end{itemize}

\vfill
\begin{center}
\rule{0.5\textwidth}{0.4pt}\\[0.3cm]
{\small REF-Manager v3.0.0 --- GNU General Public License v3.0}\\
{\small Department of Language and Linguistic Science, University of York}
\end{center}

\end{document}
